% ************************** Thesis Abstract *****************************
% Use `abstract' as an option in the document class to print only the titlepage and the abstract.
\begin{abstract}

Exoplanets orbiting distant stars have revealed to us that no one could have predicted Earth. Many stars host planets which appear to be made of rock and iron like Earth, but sit at several times its mass---we have no visible analogues nearby to show us what these massive rocky planets look like, and how differ from our own. Further, measurements of stellar photospheres show a certain variability in their rock-forming element abundances, which implies a similar spread in the rocky building blocks of planets. Consequences of a planetary mass variability and compositional variability comprise the contents of this thesis. Because the surfaces of exoplanets are not directly detectable with present technology---their nature cannot be told by astronomy alone---questions about rocky planet diversity require a theoretical approach. Drawing on the wealth of geoscientific knowledge originally developed to understand Earth, I construct physical models of other possible worlds to see how ours fits in. The first consequence of variability I model is topography. Topography sets the size of the largest ocean that planets could contain below their highest point---the capacity of the continental bathtub. I calculate the bare-minimum, dynamic topography due directly to mantle convection, by finding scaling laws of dynamic topography with convective parameters. For increasingly massive planets, topography almost disappears. Smaller ocean basins might suggest flooded worlds, all else equal. Yet large portions of a planet's water can be buried in their mantles, predictable insofar as known mantle minerals have characteristic maximum water contents. Thus for a second consequence of variability, I leverage stellar abundances to constrain the mineral phase equilibria of exoplanet mantles. I assess whether various exoplanets would plausibly sequester water in their mantles, providing a key initial condition for planetary evolution. In another chapter, I then link these surface and interior reservoirs explicitly, employing a coupled 2D mantle convection and melting model to estimate rates of volcanic outgassing on the early Earth, before modern plate tectonics---early Earth provides an elucidative case study to ground our understanding of planet diversity. These estimates focus on the possible range of mantle oxygen fugacities, a measure of how oxidising the mantle is. I show how most scenarios produce significantly lower outgassing rates in the Archean, compared to what classical thermal history models would suggest. The last consequence of variability then turns to oxygen fugacity itself. I invoke a subtle but powerful phenomenon from the petrological modelling toolbox, and predict the minimum amount by which mantle oxygen fugacity should vary across rocky exoplanets, constrained again by host star element abundances and inferred mantle mineralogy. By the end of the thesis it should become clear that there is no good prototype for a ``terrestrial planet''. Nevertheless, through the late union of exoplanet astronomers and Earth scientists, we start to appreciate how understanding planets holistically is an obligate and confrontable challenge.

\end{abstract}
